\documentclass[12pt]{article}
\usepackage[portuguese]{babel}
\usepackage{natbib}
\PassOptionsToPackage{hyphens}{url}\usepackage{hyperref}
\usepackage[utf8x]{inputenc}
\usepackage{amsmath}
\usepackage{graphicx}
\graphicspath{{images/}}
\usepackage{parskip}
\usepackage{fancyhdr}
\usepackage{vmargin}
\usepackage{hyperref}
\usepackage{indentfirst}
\hypersetup{
    colorlinks=true,
    linkcolor=black,
    filecolor=magenta,      
    urlcolor=blue,
}





%%%%%%%%%%%%%%%%%%  For code %%%%%%%%%%%%%%%%%%
\usepackage{listings}
\renewcommand{\lstlistingname}{Código}
\renewcommand{\lstlistlistingname}{Lista de \lstlistingname s}
\usepackage{color}

\definecolor{dkgreen}{rgb}{0,0.6,0}
\definecolor{gray}{rgb}{0.5,0.5,0.5}
\definecolor{mauve}{rgb}{0.58,0,0.82}

\lstset{frame=tb,
  language=Python,
  aboveskip=3mm,
  belowskip=3mm,
  showstringspaces=false,
  columns=flexible,
  basicstyle={\small\ttfamily},
  numbers=none,
  numberstyle=\tiny\color{gray},
  keywordstyle=\color{blue},
  commentstyle=\color{dkgreen},
  stringstyle=\color{mauve},
  breaklines=true,
  breakatwhitespace=true,
  tabsize=3
}
%%%%%%%%%%%%%%%%%%  For code %%%%%%%%%%%%%%%%%%

\setmarginsrb{3 cm}{2.5 cm}{3 cm}{2.5 cm}{1 cm}{1.5 cm}{1 cm}{1.5 cm}
\setlength{\parindent}{6ex}
\date{\today}											% Date

\makeatletter

\let\thedate\@date
\makeatother

\pagestyle{fancy}
\fancyhf{}
\cfoot{\thepage}

\begin{document}

%%%%%%%%%%%%%%%%%%%%%%%%%%%%%%%%%%%%%%%%%%%%%%%%%%%%%%%%%%%%%%%%%%%%%%%%%%%%%%%%%%%%%%%%%

\begin{titlepage}
	\centering
    \vspace*{0.5 cm}
    \includegraphics[scale = 0.6]{img/logo_UA.png}\\[1.0 cm]	% University Logo
    \textsc{\LARGE Universidade de Aveiro }\\[2.0 cm]
	\textsc{\Large Tecnologias e Programação Web}\\[0.5 cm]
	\rule{\linewidth}{0.2 mm} \\[0.4 cm]
	{ \huge \bfseries {Arcade Battle}}\\[0.5 cm]
	\textsc{\Large Relatório Final}\\	
	\rule{\linewidth}{0.2 mm} \\[1.5 cm]
	
	\begin{minipage}{0.4\textwidth}
		\begin{flushleft} \large
			\emph{Autores :}\\
			{84793 \quad Daniel Nunes}\\	
			{84921 \quad Rafael Direito}	
	    \end{flushleft}
	\end{minipage}
	\begin{minipage}{0.4\textwidth}
		\begin{flushright} \large
			\emph{Professor :} \\
			{Hélder Zagalo}
		\end{flushright}
	\end{minipage}\\[2 cm]
	
	\vspace*{\fill}
	{\large \thedate}
 
	\vfill
	
\end{titlepage}

%%%%%%%%%%%%%%%%%%%%%%%%%%%%%%%%%%%%%%%%%%%%%%%%%%%%%%%%%%%%%%%%%%%%%%%%%%%%%%%%%%%%%%%%%

\tableofcontents
\pagebreak

\lstlistoflistings
\pagebreak

%%%%%%%%%%%%%%%%%%%%%%%%%%%%%%%%%%%%%%%%%%%%%%%%%%%%%%%%%%%%%%%%%%%%%%%%%%%%%%%%%%%%%%%%%

\section{Introdução}
O Arcade Battle é um sistema que permite aos seus utilizadores realizarem o tratamento de artrites através de jogos lúdicos. Desta forma, utilizando uma Leap Motion, conseguimos recolher quais os gestos que um determinado utilizador está a executar, avaliá-los e utilizá-los como comandos em jogos arcade.
\par A aplicação aqui em análise surge como suporte para a gestão de todo este sistema. Nesta, os médicos podem introduzir novos pacientes no sistema, \mbox{adicionar-lhes} gestos e avaliar qual o seu desempenho ao longo de todo o tratamento.
\par A plataforma permite, também, gerir quais o médicos que podem aceder ao sistema e a adição de novos jogos ao mesmo.

\par Tendo em conta a escalabilidade da solução desenvolvida, optámos por desenvolver uma aplicação composta por 3 módulos separados, que podem ser executados em 3 ambientes distintos:

\begin{itemize}
  \item Interface Gráfica do Médico;
  \item REST API;
  \item Base de Dados.
\end{itemize}
\clearpage

%%%%%%%%%%%%%%%%%%%%%%%%%%%%%%%%%%%%%%%%%%%%%%%%%%%%%%%%%%%%%%%%%%%%%%%%%%%%%%%%%%%%%%%%%

\section{Angular}

Angular é uma plataforma open-source, liderada pela Google, de aplicações web  com front-end baseado em TypeScript. No segundo projeto, Angular foi a tecnologia utilizada para a interface e, para tal, foram implementadas as seguintes funcionalidades providenciadas pelo Angular:

\begin{itemize}
    \item \textbf{Components}: Usámos components para todas as views e lógica associada às mesmas;
    \item \textbf{Data Binding}: Data binding foi usado para, principalmente, mostrar dados persistentes;
    \item \textbf{Directives}: Usámos directives para providenciar comportamento adicional a elementos do DOM;
    \item \textbf{Service}: Todos os dados vindos do servidor foram chamados obtidos de um service;
    \item \textbf{Dependency Injection}: Injeção do service em diversos componentes, por exemplo;
    \item \textbf{Routing}: Toda a navegação dentro da plataforma é feita através de routing;
    \item \textbf{Bootstrap}: Bootstrap é utilizado como ferramenta de estilo para a interface;
    \item \textbf{Observables}: Usados, principalmente, para lidar com programação assíncrona.
\end{itemize}

\subsection{Alterações face ao primeiro projeto}
\subsubsection{Two-way Binding e Event Binding}
Enquanto que, em Django, a parte lógica associada a uma view apenas é atualizada recorrendo a métodos adicionais, em Angular, através do two-way binding, conseguimos facilmente atualizar uma variável de acordo com a view. 
\par Devido à simplicidade providenciada por este data binding, foi fortemente utilizado em casos como, por exemplo, os formulários. Os Django Forms utilizados no primeiro projeto acabaram por ser substituídos com recurso à utilização do ngSubmit (event binding) e do two-way binding, como foi referido anteriormente.
\clearpage

\begin{lstlisting}[caption={Formulário para adicionar um novo jogo (parcial)},captionpos=b]
<form (ngSubmit)="addGame(photo)" enctype="multipart/form-data">
  <div class="card-body card-block">
    <div class="row form-group">
      <div class="col col-md-1 col-sm-2 ">
        <label for="id_name">Name:</label>
      </div>
      <div class="col-12 col-md-5 col-sm-4">
        <input [(ngModel)]="name" type="text" name="name" placeholder="Insert game's name" class="form-control" required="" id="id_name">
      </div>
    </div>
\end{lstlisting}

\subsubsection{NgIf e NgFor}
No que toca a diretivas, o angular apresenta vários tipos sendo que as mais utilizados no nosso projeto foram, sem dúvida, as diretivas estruturais, mais especificamente, o NgIf e o NgFor. Estas permitem facilmente substituir as template tags provenientes do Django e renderizar de uma forma naturalmente fluída o template necessário, assim como o seu conteúdo.

\vspace{8mm}

\begin{lstlisting}[caption={Listagem de todos os médicos na plataforma},captionpos=b]
<tr *ngFor="let d of doctors">
  <td (click)="access_patient(d)">{{ d.first_name }} {{ d.last_name }}</td>
  <td class="text-center" (click)="access_patient(d)">{{ d.specialty }}</td>
  <td class="text-center" (click)="access_patient(d)">{{ d.city }}</td>
  <td class="text-center" (click)="access_patient(d)">{{ d.birth_date }}</td>
  <td class="text-center" (click)="access_patient(d)">{{ d.nif }}</td>
  <td (click)="access_patient(d)">{{ d.username }}</td>
  <td *ngIf="(userType == 'admin')">
    <button type="button" class="btn-sm btn-danger" data-toggle="modal" data-target="#deleteModal" (click)="loadInfo(d)"> Remove </button>
  </td>
</tr>
\end{lstlisting}
\clearpage

\subsubsection{Service e Dependency Injection}
Por fim, como já foi referido, todas as chamadas à REST API são feitas a partir de um service criado na aplicação Angular. Para que isto aconteça, é utilizada Dependency Injection nos componentes que precisam de dados provenientes do servidor, injetando o service nesses mesmos componentes. Através deste processo, conseguimos tornar os componentes mais sustentáveis, reusáveis e testáveis através da remoção de dependências hard coded.

\vspace{8mm}

\begin{lstlisting}[caption={Injeção de um serviço e uso posterior do mesmo},captionpos=b]
import {ArcadebattleService} from '../arcadebattle.service';

(...)

export class AllDoctorsComponent implements OnInit {
  userType: string;
  doctors: object[];
  remove: any;

  constructor(private arcadeBattleService: ArcadebattleService, private location: Location) {
    this.userType = localStorage.getItem('userType');
  }
  
  (...)
  
  getAllDoctors() {
    this.arcadeBattleService.all_doctors()
        .subscribe(
            data => {
              this.doctors = data.data;
            }
        );
    $('#dtBasicExample tr:last').remove();
  }
\end{lstlisting}
\clearpage

\section{Django REST Framework}

A Django Rest Framework (DRF) é uma biblioteca do Framework Django utilizada para a implementação de REST APIs. 
\par Esta permite-nos a utilização de políticas de autenticação e autorização, bem como serialização de dados provenientes de uma base de dados.

\subsection{Alterações face ao primeiro projeto}
\subsubsection{Autenticação}
O maior desafio da implementação do sistema utilizando uma REST API prende-se com o facto da autenticação de utilizadores ter de ser efetuada ao nível da API, em vez de na interface através da qual os utilizadores acedem ao sistema.
\par Para tal, recorremos a AuthTokens.
\par Desta forma, sempre que o utilizador envia os seus dados de login, este recebe um Token de Autenticação que deve acompanhar todos os pedidos à API que forem efetuados posteriormente.
\par O Token fica guardado do lado do cliente e é eliminado aquando o logout do mesmo, sendo, então, criado um novo token para cada sessão.

\vspace{8mm}

\begin{lstlisting}[caption={Tratamento da autenticação de um utilizador na plataforma},captionpos=b]
@csrf_exempt
@api_view(["POST"])
@permission_classes((AllowAny,))
def login(request):

    username = request.data.get("username")
    password = request.data.get("password")
    
    if username is None or password is None:
        return Response({'error': 'Please provide both username and password'}, status=HTTP_400_BAD_REQUEST)
        
    
    
    
    
    user = authenticate(username=username, password=password)
    
    if not user:
        return Response({'error': 'Invalid Credentials'}, status=HTTP_404_NOT_FOUND)
        
    token, _ = Token.objects.get_or_create(user=user)

    #Logs
    logging.info(" User: " + username +" has logged in with auth_token: " + token.key)

    u = User.objects.get(username=username)

    data = {'token': token.key,
            'user_type': get_user_type(username),
            'first_name': u.first_name,
            'last_name': u.last_name,
            'email': u.username,
            }

    try: data['photo_b64'] = Person.objects.get(user=User.objects.get(username=username)).photo_b64
    except: data['photo_b64'] = ""

    return Response( data, status=HTTP_200_OK)
\end{lstlisting}

\subsubsection{URLs e Views}
Na criação da REST API foram disponibilizados novos métodos, de forma a permitir o envio de todos os dados necessários para a visualização de informação, bem como para a introdução e atualização da mesma.
\par Desta forma, disponibilizámos métodos GET, POST e DELETE.
\par Tendo em conta a separação lógica e a melhor organização do código, criámos um novo ficheiro python (queries.py) cuja única função é a interação com a base de dados e a serialização dos dados necessários.
\par O ficheiro views.py, por sua vez, invoca métodos do queries.py, avaliando sempre se o utilizador tem autorização para efetuar o pedido em causa.

\subsubsection{Novos módulos de Python}
Para o bom funcionamento da REST API foi necessário instalar os seguintes módulos:
\begin{itemize}
    \item \textit{djangorestframework}
    \item \textit{django-cors-headers}
    \item \textit{mysqlclient}
\end{itemize}

\subsubsection{Acesso a Bases de Dados Remotas}
A REST API desenvolvida está apta para interagir com bases de dados remotas.
Uma vez que a base de dados remota utilizada era uma base de dados MySQL, foi necessário instalar o módulo mysqlclient para que fosse possível interagir com ela.
\par Isto permite-nos facilmente criar uma aplicação de 3 camadas o que traz claras vantagens para a escalabilidade da mesma.
\par Seguem-se as configurações necessárias para permitir o acesso à base de dados remota. Estas devem ser feitas no ficheiro \textit{settings.py}:

\vspace{8mm}

\begin{lstlisting}[caption={Configurações para acesso à base de dados remota},captionpos=b]
DATABASES = {
  'default': {
    'ENGINE': 'django.db.backends.mysql',
    'NAME': 'arcadebattle_db',
    'USER': "arcadebattle",
    'PASSWORD': "arcadebattle",
    'HOST': "arcadebattle-db.cysle9zhxbja.us-east-1.rds.amazonaws.com",
    'PORT': "3306",
  }
}
\end{lstlisting}
\clearpage

\section{Notas Finais}

Git: \url{https://github.com/danielfbnunes/arcadebattle_angularapp} \\
\par REST Api: \url{http://rdireito.pythonanywhere.com} \\
\par Interface do médico: \url{http://ec2-100-24-34-88.compute-1.amazonaws.com/}

\vspace{8mm}

Olhando em retrospectiva, consideramos que todos os objetivos relativos a este trabalho foram atingidos. Como nota pessoal, achamos que, sem sombra de dúvidas, django é sem dúvida uma tecnologia muito apetecível para este tipo de trabalhos sendo que o angular se torna um pouco mais trabalhoso.
\par Em suma, concluímos que, com este trabalho, os conhecimentos de django e angular foram, sem qualquer dúvida, melhorados e que a capacidade de pesquisa e autonomia foram, também elas, bem desenvolvidas, na tentativa de obter soluções face aos problemas encontrados. É ainda de realçar que, com esta cadeira, aprendemos tecnologias que certamente serão relevantes no nosso percurso profissional.

\end{document}
